\documentclass[a4paper, 10pt, conference]{ieeeconf}

\IEEEoverridecommandlockouts
\overrideIEEEmargins

\usepackage{graphics} % for pdf, bitmapped graphics files
\usepackage{epsfig} % for postscript graphics files
\usepackage{mathptmx} % assumes new font selection scheme installed
\usepackage{times} % assumes new font selection scheme installed
\usepackage{amsmath} % assumes amsmath package installed
\usepackage{amssymb}  % assumes amsmath package installed
\usepackage{hyperref}
\usepackage{listings}
\usepackage{subfigure}

\usepackage[T1]{fontenc}
\usepackage[utf8]{inputenc}
\usepackage[brazil]{babel}
\usepackage{etoolbox}
\patchcmd{\abstract}{Abstract}{Resumo}{}{}
\patchcmd{\thebibliography}{References}{Referências}{}{}

\title{\LARGE \bf Otimizando o tempo de execução no processamento de imagens}

\author{Henrique Miyamoto e Thiago Benites \thanks{* Os códigos do projeto estão disponíveis em \url{https://github.com/miyamotohk/linguagem-processamento-imagem}.}}

\begin{document}
\maketitle
\thispagestyle{empty}
\pagestyle{empty}

%\begin{abstract}

%Escreva aqui o resumo (abstract).

%\end{abstract}

\section{Contextualização}

%Um breve texto introdutório explicando do que se trata o documento, em uma linguagem que poderia ser entendida por qualquer pessoa que entenda programação (ou seja: referências diretas à disciplina não são desejáveis).

Apresentamos uma comparação entre diferentes métodos para aplicação de brilho em nossa linguagem voltada para o processamento de imagens. A mesma função foi implementada de quatro diferentes maneiras: usando múltiplas \textit{threadas}, usando multiprocessos, em uma única linha de execução, varrendo a matriz por linhas e por colunas. O objetivo desse trabalho é comparar e discutir os desempenhos de cada método. Para isso, foi medido o tempo de execução de cada implementação.

A tabela \ref{tabela1} apresenta as sintaxes para os diferentes métodos de aplicação de brilho usados em nosso programa.

\begin{table}[h]
	\centering
	\caption{Sintaxe da linguagem de programação para diferentes métodos}
	\label{tabela1}
	\begin{tabular}{|c|c|}
		\hline \textbf{Funcionalidade} & \textbf{Sintaxe}\\
		\hline Alterar brilho usando \textit{multithreadas} & \begin{tabular}[c]{@{}c@{}}{destino.jpg = origem.jpg * float THR}\end{tabular} \\
		\hline Alterar brilho usando multiprocessos & \begin{tabular}[c]{@{}c@{}}{destino.jpg = origem.jpg * float PRC}\end{tabular} \\ 
		\hline Alterar brilho com varredura por linhas & \begin{tabular}[c]{@{}c@{}}{destino.jpg = origem.jpg * float LIN}\end{tabular} \\ 
		\hline Alterar brilho com varredura por colunas & \begin{tabular}[c]{@{}c@{}}{destino.jpg = origem.jpg * float COL}\end{tabular} \\ 
		\hline
	\end{tabular}
\end{table}

Neste caso, a implementação simples se dá pela varredura, pixel a pixel, da matriz que representa a imagem. Já a implementação com \textit{multithreads}, faz com que as operações de alteração do brilho sejam feitas em grupos de pixel e de forma paralela.

\section{Demonstração}

%Entradas e saídas que demonstram as funcionalidades implementadas.

Para medição dos tempos de execução foi usada a biblioteca \textit{time.h}. A tabela \ref{tabela2} indica os tempos de processamento para a aplicação da mesma intensidade de brilho em imagens de diferentes tamanhos. Também alterou-se a quantidade de pixels que são tratados em cada \textit{thread} executada.

\begin{table}[h]
	\centering
	\caption{Tempo de processamento de acordo com a implementação e tamanho da imagem}
	\label{tabela2}
	\begin{tabular}{|c|c|c|}
		\hline
		\textbf{Implementação}        & \textbf{Tamanho}                                                                                                               & \textbf{Tempo} \\ \hline
		Simples     & {48x48}	& {0,046 ms}                                                                                    \\ \hline
		Multithread (10 pixels)        & {48x48}	& {3,31 ms} \\ \hline
		Multithread (48 pixels)        & {48x48}	& {0,45 ms} \\ \hline
		Simples     & {2592x1944}	& {136,63 ms}                                                                                    \\ \hline
		Multithread (600 pixels)        & {2592x1944}	& {69,87 ms} \\ \hline
		Multithread (2592 pixels)        & {2592x1944}	& {14,99 ms} \\ \hline
		
	\end{tabular}
\end{table}

\section{Análise}

%Aqui deve-se comparar a qualidade, em questão de tempo, de cada implementação. Ademais, analisar-se-ão quais métodos são vantajosos de acordo com diferentes tamanhos de imagens e de pixels fixados para análise por função.

Observa-se que a execução da função de brilho através de \textit{multithreading} não apresenta desempenho estritamente superior ao com varredura simples. Para imagens pequenas, o processamento simples é mais rápido que com \textit{multithread}. Por outro lado, quando as imagens são grandes, usar \textit{multithreads} aumenta consideravelmente a velocidade de execução da função. Em ambos os casos, aumentar a quantidade de pixels tratados em cada \textit{thread} melhora o desempenho.

\begin{thebibliography}{99}

\bibitem{dsl} MERNIK, M., HEERIN, J., SLOANE, A. M. When and how to develop domain-specific languages. In: \textit{ACM Computing Surveys (CSUR)}. Nova York, vol. 37, ed. 4, p. 316-344, dez. 2005.

\end{thebibliography}

\end{document}