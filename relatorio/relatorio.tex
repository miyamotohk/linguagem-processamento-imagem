\documentclass[a4paper, 10pt, conference]{ieeeconf}

\IEEEoverridecommandlockouts
\overrideIEEEmargins

\usepackage{graphics} % for pdf, bitmapped graphics files
\usepackage{epsfig} % for postscript graphics files
\usepackage{mathptmx} % assumes new font selection scheme installed
\usepackage{times} % assumes new font selection scheme installed
\usepackage{amsmath} % assumes amsmath package installed
\usepackage{amssymb}  % assumes amsmath package installed
\usepackage{hyperref}
\usepackage{listings}

\usepackage[T1]{fontenc}
\usepackage[utf8]{inputenc}
\usepackage[portuges]{babel}
\usepackage{etoolbox}
\patchcmd{\abstract}{Abstract}{Resumo}{}{}
\patchcmd{\thebibliography}{References}{Referências}{}{}

\title{\LARGE \bf Uma linguagem de programação para processamento de imagens}

\author{Henrique Miyamoto e Thiago Benites}


\begin{document}
\maketitle
\thispagestyle{empty}
\pagestyle{empty}

%\begin{abstract}

%Escreva aqui o resumo (abstract).

%\end{abstract}

\section{Contextualização}

%Um breve texto introdutório explicando do que se trata o documento, em uma linguagem que poderia ser entendida por qualquer pessoa que entenda programação (ou seja: referências diretas à disciplina não são desejáveis).

Apresentamos uma linguagem de programação voltada para o processamento de imagens, implementada com o analisador léxico Flex \cite{flex} e o compilador de compilador GNU Bison\cite{bison}. As funções que ela é capaz de executar, bem como as respectivas sintaxes são apresentadas na Tabela \ref{tabela1}.

\begin{lstlisting}[language=C, basicstyle=\footnotesize, frame=single]
int main(){
	coloque_o_seu(codigo, aqui);
}
\end{lstlisting}

\section{Demonstração}

Entradas e saídas que demonstram as funcionalidades implementadas.

\section{Análise}

Comparando a idéia de usar os comandos específicos da linguagem para aplicar brilho com uma aplicação equivalente, usando alguma biblioteca de linguagem de propósito geral (por exemplo, `OpenCV`). A análise deve se basear em dados reais, e mostrar todos os dados sobre os quais ela se baseia (exemplos de código, citações bibliográficas ou outros dados que o grupo considere relevantes).

\begin{thebibliography}{99}

\bibitem{flex} GitHub. The Fast Lexycal Analyzer. Disponível em: \url{https://github.com/westes/flex}. Acesso em: 9 set. 2017. 

\bibitem{bison} GNU Bison. Disponível em: \url{http://www.gnu.org/software/bison/}. Acesso em: 9 set. 2017.

\end{thebibliography}

\end{document}